
\section{cDNA-Synthese}
Die Synthese von \acs{cDNA} aus pflanzlicher und pilzlicher RNA erfolgte mit dem Tetro cDNA Synthese Kit (Bioline GmbH, Luckenwalde). Dafür wurden bis zu 5\,\mug\ total RNA eingesetzt. Um mögliche Kontaminationen mit genomischer DNA (\acs{gDNA}) zu eliminieren, wurde die verwendete RNA einem DNAse\,-\,Verdau mit PerfeCTa~DNAse I (Quanat Bioscience, Gaithersburg, MD, USA) unterzogen. Anschließend erfolgte die  Synthese von cDNA unter Verwendung von Hexamer-Primern. 
\begin{minipage}[t]{.5\textwidth}
\captionof{table}[Reaktionsansatz für den Verdau mit DNAse\,I]{Reaktionsansatz für den Verdau mit DNAse\,I}
\begin{tabular}{
p{.75\textwidth}
S[table-format=1.0,table-comparator=true,table-space-text-post={***}]
}\label{tab:DNAse-Verdau}
\toprule
Reagenz & \si{Volumen(\mul)}\\
\midrule
10x-Reationspuffer&1\mul \\
DNAse\,I&1\,\mul \\
RNA & bis 5\,\mug \\
\reinstwasser & auf 10\,\mul \\
\multicolumn{2}{l}{Der Reaktionsansatz wurde für 30\, min bei 37\celcius\ inkubiert. Anschließend wurde 1\,\mul\ Inaktivierungspuffer zugegeben und der Reaktionsansatz für 10\,min bei 65\celcius\ inkubiert. }
\bottomrule
\end{tabular}
\end{minipage}
%%%
\begin{minipage}[t]{.5\textwidth}
\captionof{table}[Reaktionsansatz für die cDNA-Synthese]{Reaktionsansatz für die cDNA-Synthese}
\begin{tabular}{
p{.75\textwidth}
S[table-format=1.0,table-comparator=true,table-space-text-post={***}]
}
\toprule
Reagenz & \si{Volumen(\mul)}\\
\midrule
10x-Reationspuffer&1\mul \\
dNTPS (10mM)&1\,\mul \\
Hexamer-Primer(10mM)&1\,\mul \\
RNA (aus \ref{tab:DNAse-Verdau}  & 11\mul \\
RNAse-Inhibitor & 1\,\mul \\
Reverse Transkriptase & 1\,\mul \\ 
\reinstwasser & auf 20\,\mul \\ 
\bottomrule
\end{tabular}
\end{minipage}
