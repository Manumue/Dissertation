\section{Isolierung von Nukleinsäuren}
\subsection{Isolation von Plasmid DNA}
Die Isolation von Plasmid-DNA aus \Ecoli\ und \Atumefaciens\ erfolgte mittels peqGold Plasmid Mini Kit (Peqlab GmbH, Erlangen) nach den Angaben des Herstellers. 
\subsection{Isolation von RNA aus Pflanzenmaterial}\label{sec:RNA-Pflanze}
Die Isolation von RNA aus infiziertem und nicht-infiziertem Pflanzenmaterial von \Gmax\ erfolgte nach einem modifizierten Protokoll mittels Plant RNA Isolation Kit (Agilent GmbH, Böblingen). Dafür wurden bis zu 100\, mg Pflanzenmaterial mit einem Korkbohrer ausgestanzt und zusammen mit zwei Edelstahlkügelchen (\O 4\,mm) in 2\, ml Schraubdeckelröhrchen überführt. Das Pflanzenmaterial wurde in flüssigem Stickstoff schockgefroren und zweimal für 20\,s bei 4000 \acs{mps} im FastPrep$\textsuperscript{\textregistered}$-24 homogenisiert, wobei das Material zwischen den Homogenisierungsschritten erneut schockgefroren wurde. Anschließend wurden 600\,$\upmu$ Extraktionslösung dazugegeben und durch vortexen gemischt. Das Homogenat wurde für 2\,min bei 4\celcius\ und 16.000\,rpm zentrifugiert. Anschließend wurde der Überstand abgenommen und auf ein Filtersäulchen überführt. Nach einer 3-minütigen Zentrifugation bei 4\celcius\ und und 16.000\,rpm wurde der Durchfluß in ein RNAse-freies 2\,ml Reaktionsgefäß überführt und 600\,$\upmu$l Isopropanol dazugegeben. Die Lösung wurde durch mehrfaches invertieren gemischt und für 5\,min bei Raumtemperatur inkubiert. Anschließend wurden 600\,$\upmu$l der Lösung auf ein Isolationssäulchen überführt und für 30\,s bei  4\celcius\ und 16.000\,rpm zentrifugiert. Der Durchfluss wurde verworfen und das Säulchen erneut mit 600 $\upmu$l beladen und ein weiteres mal für 30\,s bei  4\celcius\ und 16.000\,rpm zentrifugiert. Es folgten 2 Waschschritte bei welchen jeweils 500\,$\upmu$l Waschlösung auf das Säulchen gegebebn wurden und anschließend für 30 s bei  4\celcius\ und 16.000\,rpm zentrifugiert wurde. Der Durchfluss wurde verworfen und das Säulchen zur Trocknung der gebundenen RNA für 2\,min bei 4\celcius\ und 16.000\,rpm zentrifugiert. Anschließend wurden 30-50\,$\upmu$l \depcwasser\ auf die Mitte der Säulchenmembran pipettiert und das Säulchen auf ein RNAse-freies 1,5\,ml Reaktionsgefäß überführt. Nach einer Inkubationszeit von 2\,min bei Raumtemperatur wurde die gelöste RNA durch Zentrifugation für 30 s bei  4\celcius\ und 16.000\,rpm in das Reaktionsgefäß überführt. Die kurzzeitige Lagerung der RNA erfolgte auf Eis. Zur langfristigen Lagerung wurde die RNA durch Zugabe von 1/10 Volumen Natriumacetat (3M, pH irgendwas) und 2,5 Volumen EtOH gefällt und bei -80\celcius\ aufbewahrt. 
\subsection{Isolation von RNA aus Uredosporen und Keimschläuchen von \Ppach}
Die Isolation von RNA aus Uredosporen und Keimschläuchen (siehe \ref{sec:Keimschlauch} erfolgte analog zu der in \ref{sec:RNA-Pflanze} beschriebenen Methode.    
\subsection{Isolation von RNA aus Appressorien von \Ppach}
Zur Isolation von RNA aus Appressorien von \Ppach wurden die in\ref{sec:appressorien} beschriebenen PE-Folie behutsam mit Filterpapier trocken getupft und anschließend 600\,$\upmu$l Extraktionslösung darauf gegebenen. Die Extraktionslösung wurde mit einem Gummiwischer (Carl Roth GmbH, Karlsruhe} verteilt, wodurch sich die pilzlichen Strukturen von der Folie lösten und mit einer abgeschnittenen Pipettenspitze in ein 2\,ml Reaktionsgefäß mit einer Mischung aus Quarzsand und Glaskügelchen (Lysing Matrix E, MP Biomedicals GmbH, Eschwege) überführt werden konnten. Die Appressorien wurden 2-mal für 20\,s bei 6500 \acs{mps} im FastPrep$\textsuperscript{\textregistered}$-24 homogenisiert, wobei das Material zwischen den Homogenisierungsschritten auf Eis gekühlt wurde. Das Homogenat wurde für 2\,min bei 4\celcius\ und 16.000\,rpm zentrifugiert. Anschließend wurde der Überstand abgenommen und auf ein Filtersäulchen überführt. Die weitere Vorgehensweise erfolgte analog zu der in \ref{sec:RNA-Pflanze} beschriebenen Methode. 