\documentclass[ toc=listof,			%Abbildungs- und Tabellenverzeichnis ins Inhaltsverzeichnis
 bibtotoc, %Literaturverzeichnis ins Inhaltsverzeichnis
  listof=entryprefix, %Tab und Fig ins Tabellen und Abbildungsverzeichnis
  12pt, %Schriftgröße
  captions=nooneline,
  tablecaptionabove
 ] %Tabellenüberschrift/abbildungsunterschrift linksbündig
  {scrreprt} 								% Dokumentklasse
%\usepackage[skip=30pt]{caption}
\usepackage[ngerman]{babel} 									% Neue deutsche Rechtschreibung
\usepackage[T1]{fontenc} 											% Ausgabezeichen
\usepackage[utf8]{inputenc} 										% europäische Zeichenkodierung und Umlaute
\usepackage[left=2.5cm,right=2.5cm,top=1cm,bottom=1cm,includeheadfoot]{geometry}	% Seitenränder, links inkl. 1cm 
\usepackage{setspace} 
%Bindekorrektur
%\renewcommand{\familydefault}{\sfdefault}				% Umstellung auf serifenlose Schrift 
%\usepackage{mathpazo}
%\usepackage{helvet}												% Schriftart Helvetica
%\usepackage{fancyref}
%\usepackage{tocloft}
%\usepackage{lmodern}
%%Pakete allgemein
\usepackage{chngcntr} 					%		Nummerierung der Tabellen und Abbildungen in normalem Format
\counterwithout{footnote}{chapter}   % 	Fußnoten fortlaufend 
\counterwithout{figure}{chapter}      % 	Abbildung ohne Kapitelnummer. 
\counterwithout{table}{chapter}         % 	Tabellen ohne Kapitelnummer.. 
\usepackage{palatino}
\usepackage{blindtext}
%\usepackage{lmodern}                     %		Schriftart lmodern
%\renewcommand{\familydefault}{\sfdefault}											% Umstellung auf serifenlose Schrift 
%\usepackage{mathpazo}
%\usepackage{helvet}																				% Aktivierung von 1,5-fachem Zeilenabstand
\setcounter{secnumdepth}{4}															% Nummerierung bis zur 4. Gliederungsebene (Subsubsection)
\setcounter{tocdepth}{4}
\newcommand{\chapternumbering}[1]{% 												% Wechsel zwischen römischen und arabischen Ziffern bei Kapitelnummerierung 
  \setcounter{chapter}{0}% 															%
   \renewcommand{\thechapter}{\csname #1\endcsname{chapter}}} 						%
\renewcommand*{\chapterheadstartvskip}{\vspace*{-20pt}}								% Abstand zwischen Kapitelnummer und Seitenrand 
\usepackage{natbib}																	% Literaturverzeichnis
\usepackage{booktabs}																% Tabellen
%\usepackage{longtable,ltcaption}
															% Tabellen über mehrere Seiten
\usepackage{tabularx}																	%Tabellen mit automatisch ermittelter Spaltenbreite
%\usepackage{tabularx}
\newcolumntype{L}[1]{>{\raggedright\arraybackslash}p{#1}} % linksbündig mit Breitenangabe
\newcolumntype{C}[1]{>{\centering\arraybackslash}p{#1}} % zentriert mit Breitenangabe
\newcolumntype{R}[1]{>{\raggedleft\arraybackslash}p{#1}} % rechtsbündig mit Breitenangabe
\usepackage{longtable,ltcaption}
\newcommand{\ltab}{\raggedright\arraybackslash} % Tabellenabschnitt linksbündig
\newcommand{\ctab}{\centering\arraybackslash} % Tabellenabschnitt zentriert
\newcommand{\rtab}{\raggedleft\arraybackslash} % Tabellenabschnitt rechtsbündig 
\usepackage{capt-of}																	%Beschriftung Abb. und Tab. ohne Gleitumgebung
%\usepackage{float}										
\usepackage{graphicx}																% Abbildungen einfügen
\usepackage{subfig}																	% Mehrere Abbildungen nebeneinander 
\usepackage[plain]{fancyref}													% Querverweise im Text, ohne Seitenzahlen			
\usepackage{mdwlist}																% Kleinere Abstände zwischen einzelenen Items in einer Liste
\usepackage{siunitx}																% richtige Darstellung von SI-Einheiten
\sisetup{%
  mode = math,
  detect-all,
  detect-weight,  
  exponent-product = \cdot,
  number-unit-separator=\text{\,},
  output-decimal-marker={\text{,}},
  math-rm=\sffamily,
  text-rm=\rmfamily,
}													% ", " als Dezimaltrennzeichen
%\sisetup{ 
  %locale=DE 
%} 
\usepackage{upgreek}
\usepackage[printonlyused]{acronym}										% Abkürzungsverzeichnis
\usepackage{epigraph}																% Zitate am Anfang der Arbeit
\renewcommand{\epigraphflush}{center}												% Zitat in Seitenmitte
\renewcommand{\epigraphsize}{\normalsize}											% Zitat in normaler Textgröße
\usepackage{scrpage2}															% Paket für Kopfzeilen
\automark{chapter}																	% laufender Kolumnentitel
\clearscrheadfoot
\ihead{}
\chead{}
\ohead{\headmark}																	% Kapitel im Kopf rechts
\ifoot{}
\cfoot[\pagemark]{\pagemark}													% Seitenzahl zentriert
\ofoot{}
\setheadsepline{0.5pt}															 	% Dicke der Linie unter Kopfzeile
\addtokomafont{pageheadfoot}{\upshape}							% Kolumnentitel normal (nicht fett oder kursiv)

%%%%%%%%%%%%%%% Eigene Labels

\newcommand{\water}{H$_2$O}
\newcommand{\celcius}{$^\circ$C}
\newcommand{\Gas}{CO$_2$}
\newcommand{\vewasser}{H$_2$O$_{VE}$}
\newcommand{\reinstwasser}{H$_2$O$_{reinst}$}
\newcommand{\BamHI}{\textit{Bam}HI}
\newcommand{\EcoRI}{\textit{Eco}RI}
\newcommand{\ClaI}{\textit{Cla}I}
\newcommand{\KpnI}{\textit{Kpn}I}
\newcommand{\XbaI}{\textit{Xba}I}
\newcommand{\XhoI}{\textit{Xho}I}
\newcommand{\Pp}{\textit{Phakopsora~pachyrhizi}}
\newcommand{\TaqMan}{TaqMan\textsuperscript{\texttrademark}}
\newcommand{\Ppach}{\textit{P.~pachyrhizi}}
\newcommand{\Gmax}{\textit{G.~max}}
\newcommand{\Nbenthamiana}{\textit{N.~benthamiana}}
\newcommand{\Atumefaciens}{\textit{A.~tumefaciens}}
\newcommand{\Ecoli}{\textit{E.~coli}}
\newcommand{\depcwasser}{H$_2$O$_{DEPC}$}
%%%%%%%%%%%%%%%%%%
\AtBeginDocument{% 
  \renewcaptionname{ngerman}{\figurename}{Abb.}% scrguide, Abschnitt 10.4 
                                % (Seite 265f - einschl. Beispiel) 
  \newcaptionname{ngerman}{\listoflofentryname}{Abb.}% scrguide, 
                                % Abschnitt 3.20 (Seite 136) 
  \renewcaptionname{ngerman}{\tablename}{Tab.}% scrguide, Abschnitt 10.4 
                                % (Seite 265ff) 
  \newcaptionname{ngerman}{\listoflotentryname}{Tab.}% scrguide, 
                                % Abschnitt 3.20 (Seite 136) 
} 
\setuptoc{toc}{totoc}
\begin{document} 
\titlehead{% 
\flushright
\includegraphics[scale=1.25]{Deckblatt/logo1}}
\subject{Aus dem Institut für Phytomedizin \\Fachgebiet Phytopathologie\\Prof.\,Dr.\,Ralf Vögele}
\title{Etablierung eines Wirts-induzierten RNAi-Systems für die Kontrolle des Asiatischen Sojabohnenrostes\\ \textit{Phakopsora pachyrhizi} }
\subtitle{
\vspace*{.5cm}
Dissertation\\ zur Erlangung des Grades\\eines Doktors der Agrarwissenschaften (Dr.\,sc.\,agr.)\\}

\author{
Vorgelegt der Fakultät Agrarwissenschaften\\ an der Universität Hohenheim\\}

\date{
Von\\ Manuel Müller\\(Dipl.\,Agr.\,Biol.)\\
\vspace*{.5cm}
2014}

\maketitle					%Deckblatt einbinden
\include{Leerseite/Leerseite}					%Leerseite einbinden
%\include{Zitat/Zitat}									%Zitat einbinden	
%\setcounter{secnumdepth}{4} 
%\renewcommand\thechapter{\Roman{chapter}} 
\pagenumbering{roman} 
\tableofcontents
\listoffigures 
\listoftables 
\chapter*{Abkürzungsverzeichnis}
\renewcommand{\bflabel}[1]{\normalfont{\normalsize{#1}}\hfill}
\addcontentsline{toc}{chapter}{Abkürzungsverzeichnis}
\begin{acronym}[blablablablablabla]
%A
%B
%C
\acro{cDNA}{engl. complementary DNA, komplementäre DNA}
%D
\acro{DEPC}{Diethylpyrocarbonat}
\acro{DMSO}{Dimethylsulfoxid}
\acro{DNA}{engl. Deoxyribonucleic acid, Deoxyribonukleinsäure}
%E
\acro{EDTA}{Ethylendiamintetraessigsäure}
\acro{EtOH}{Ethanol}
%F
\acro{FAM}{6-Carboxyfluorescein}
%G
\acro{gDNA}{genomische DNA}
\acro{GUS}{$\beta$-Glucuronidase}
%H
\acro{\reinstwasser}{Reinstwasser}
\acro{\vewasser}{Vollentsalztes Wasser}
\acro{HEPES}{2-(4-(2-Hydroxyethyl)-1-piperazinyl)-ethansulfonsäure}
\acro{HPLC}{engl. high performance liquid chromatography}
%I
%J
%K
%L
\acro{LB}{engl. lysogeny broth}
%M
\acro{MES}{2-N-(Morpholino)ethansulfonsäure}
\acro{MOPS}{3-(N-Morpholino)-Propansulfonsäure}
\acro{mps}{engl. movements per second, Bewegungen pro Sekunde}
%N
%O
\acro{OD$_{600}$}{Optische Dichte bei einer Wellenlänge von 600\,nm}
%P
\acro{PE}{Polyethylen}
\acro{PCR}{engl. polymerase chain reaction}
%Q
%R
\acro{rcf}{engl. relative centrifugal force, relative Zentrifugalbeschleunigung}
\acro{rpm}{engl. rounds per minute, Umdrehungen pro Minute}
%S
\acro{SEM}{engl. simple and efficient method}
\acro{SOC}{engl. super optimal broth with catabolite repression}
%T
\acro{TAE}{TRIS-Acetat-EDTA}
\acro{TAMRA}{Tetramethylrhodamin}
\acro{TB}{engl. terrific broth}
\acro{TBE}{TRIS-Borat-EDTA}
\acro{TRIS}{Tris(hydroxymethyl)-aminomethan}
%U
%V
%W
%X
\acro{X-Gluc}{Cyclohexylammoniumsalz der 5-Brom-4-chlor-3-indolyl-$\beta$-D-glucuronsäure}
%Y
%Z
\acro{Matze}{Stinktier}




\end{acronym}
%\renewcommand\thechapter{\Arabic{chapter}} 
\renewcommand\thechapter{\arabic{chapter}}
\setcounter{chapter}{0}
\pagestyle{scrheadings}
\pagenumbering{arabic}
\onehalfspacing
\chapter{Einleitung}
\section{Rostpilze}
Rostpilze blabla \citep{Link2014}
\subsection{A}
Test \ac{YTM} blablabla \ac{YTM}
\subsubsection{B}
\chapter{Material und Methoden}
\section{Chemikalien}
Die in dieser Arbeit verwendeten Chemikalien waren von analytischem Reinheitsgrad und wurden, soweit im Text nicht anders angegeben, von folgenden Herstellern bezogen:\\
%\vspace{5mm}\\
\begin{tabular}[H]{p{.25\textwidth}l}
AppliChem&AppliChem GmbH, Darmstadt\\
Fluka&Sigma-Aldrich Chemie GmbH, Taufkirchen\\
Merck&Merck KGaA, Darmstadt\\
Riedel-deHa\"en&Sigma-Aldrich Chemie GmbH, Taufkirchen\\
Roth&Carl Roth GmbH \& Co. KG, Karlsruhe\\
Serva&Serva Electrophoresis GmbH, Heidelberg\\
SIGMA&Sigma-Aldrich Chemie GmbH, Steinheim\\ 
\end{tabular}
\subsection{Antibiotika und Naturstoffe}
Für die in dieser Arbeit verwendeten Antibiotika und Naturstoffe wurden Stammlösungen angesetzt, welche bis zu ihrer Verwendung bei -20\celcius\ gelagert wurden. Die Konzentration der Stammlösungen sowie das jeweilige Lösungsmittel sind in \Fref{tab:AB} aufgelistet.\\
%\begin{table}[H]
\captionof{table}[Verwendete Antibiotika und Naturstoffe]{Liste der verwendeten Antibiotika und Naturstoffe}
\label{tab:AB}
\begin{tabular}
{
p{.2\textwidth}
S[table-format=1.0,table-comparator=true,table-space-text-post={**********************}]
p{.45\textwidth}
}
\toprule
\multicolumn{1}{l}{Substanz}&\multicolumn{1}{l}{Konzentration} & \multicolumn{1}{l}{Hersteller}\\ 
\midrule
Acetosyringon & 200\,\si{mM}~in~ \acs{EtOH}& Carl Roth GmbH, Karlsruhe \\ 
Ampicilin & 100\,\si{mg\per ml}~ in~ \acs{\reinstwasser} & AppliChem GmbH, Darmstadt \\ 
Kanamycin & 50\,\si{mg\per ml} ~in~\acs{\reinstwasser} & Duchefa B.V, Haarlem, NL  \\ 
Spectinomycin & 100\,\si{mg\per ml}~ in~ \acs{\reinstwasser} & Sigma-Aldrich GmbH, Steinheim \\ 
Rifampicin & 50\,\si{mg\per ml} ~in~ DMSO & Duchefa B.V, Haarlem, NL \\ 
\bottomrule
\end{tabular}
%\end{table}
\vspace{12pt}\\
Soweit im Text nicht anders angegeben, wurden Ampicilin und Spectinomycin in einer Endkonzentration von 100\,$\upmu$g/ml, Kanamycin und Rifampicin in einer Endkonzentration von 50\,$\upmu$g/ml eingesetzt. Acetosyringon wurde in einer Endkonzentration von 200\,$\upmu$M verwendet. 
\newpage
%\section{Nährmedien, Puffer und Lösungen}
\subsection{Nährmedien}
%\vspace{12pt}
%Alle Nährmedien wurden mit \acs {\vewasser} angesetzt und anschließend für 20\,min bei 120\celcius~autoklaviert. Für die Herstellung von festen Nährmedien wurden die Medien vor dem Autoklavieren mit 1,7\si{\%}(w/v) Agar ergänzt. 
%\begin{table}[H]
\begin{longtable}{
p{.3\textwidth}
p{.3\textwidth}
S[table-format=1.3,table-comparator=true,table-space-text-post={*****}]
p{.3\textwidth}
}
\acs{LB}-Medium & Bacto-Trypton & 1\%(w/v) &  \\ 
 & Hefeextrakt & 0,5\si{\%}(w/v) &  \\ 
 & NaCl & 1\si{\%}(w/v) &  \\ 
 & pH & 7,0 &  \\ 
 &  &  &  \\ 
\acs{TB}-Medium & Bacto-Trypton & 1\si{\%}(w/v) &  \\ 
 & Hefeextrakt & 2,4\si{\%}(w/v) &  \\ 
 & Glycerin & 0,4\si{\%}(v/v) &  \\ 
 \begin{tabular}{
p{.285\textwidth}
p{.5\textwidth}
%p{1cm}
S[table-format=1.3,table-comparator=true,table-space-text-post={*****}]
p{.3\textwidth}}
&Das Medium wurde autoklaviert und  abgekühlt. Anschließend wurden 10\,\si{\%}\ (v/v) KPO$_4$-Puffer zugegeben. Die weitere Lagerung erfolgte bei Raumtemperatur.
\end{tabular}\\
&&&\\ 
SOB-Medium& Bacto-Trypton& 2\si{\%}(w/v) &  \\ 
 & Hefeextrakt & 0,5\si{\%}(w/v) &  \\ 
 &KCL& 2,5\si{mM}&\\
 & NaCl & 10\si{mM} &  \\
   \begin{tabular}{
p{.285\textwidth}
p{.5\textwidth}
%p{1cm}
S[table-format=1.3,table-comparator=true,table-space-text-post={*****}]
p{.3\textwidth}}
&Das Medium wurde autoklaviert und bei Raumtemperatur gelagert. Direkt vor der Verwendung, wurde MgSO$_4$ zu einer Endkonzentration von 10\,\si{mM}  zugegeben.&&\\
\end{tabular}\\
&&&\\
\acs{SOC}-Medium & Bacto-Trypton & 2\si{\%}(w/v) &  \\ 
 & Hefeextrakt & 0,5\si{\%}(w/v) &  \\ 
 & NaCl & 10\si{mM} &  \\ 
 & KCl & 2,5\si{mM} &  \\ 
 & MgCl$_2$ & 10\si{mM} &  \\ 
 & MgSO$_4$ & 10\si{mM} &  \\ 
   \begin{tabular}{
p{.285\textwidth}
p{.5\textwidth}
%p{1cm}
S[table-format=1.3,table-comparator=true,table-space-text-post={*****}]
p{.3\textwidth}}
&Das Medium wurde autoklaviert und  abgekühlt. Anschließend wurde Glukose zu einer Endkonzentration von 20\,\si{mM}\ zugegeben. Die weitere Lagerung erfolgte bei Raumtemperatur.
\end{tabular}\\
&&&\\ 
\newpage 
 YEB-Medium & Pepton & 5\si{\%}(w/v) &  \\ 
 & Hefeextrakt & 1\si{\%}(w/v) &  \\ 
 & Fleischextrakt & 5\si{\%}(w/v) &  \\ 
 & Saccharose & 5\si{\%}(w/v) &  \\ 
 & MgSO$_4$ & 2\si{mM} &  \\ 
 & pH & 7,2 &  \\ 
\end{longtable}
\addtocounter{table}{-1}
\subsection{Puffer und Lösungen}
    \begin{longtable}{ 
 p{.3\textwidth}
p{.3\textwidth}
%p{1cm}
S[table-format=1.3,table-comparator=true,table-space-text-post={*****}]
p{.3\textwidth}}
Anilinblau& Anilinblau& 0,05\si{\%}\ (w/v)& \\
&Glyzerin& 50\si{\%}\ (v/v)&\\
&Milchsäure&25 \si{\%}\ (v/v)&\\
&&&\\
EDTA&EDTA&0,5\si{M}&\\
&pH (NaOH)& 8,0&\\
\begin{tabular}{
p{.29\textwidth}
p{.5\textwidth}
%p{1cm}
S[table-format=1.3,table-comparator=true,table-space-text-post={*****}]
p{.3\textwidth}}
&Die Lösung wurde autoklaviert und bei Raumtemperatur gelagert.&&\\
\end{tabular}\\
&&&\\
\acs{GUS}-Färbepuffer & NaPO$_4$-Puffer (1\si{M}) & 50\si{mM} &   \\ 
& \acs{EDTA} & 10\si{mM} &  \\ 
& K$_3$Fe(CN)$_6$ & 0,5\si{mM} &  \\ 
& K$_4$Fe(CN)$_6$ & 0,5\si{mM} &  \\ 
& Triton X-100 & 0,1\si{\%}\ (v/v) &  \\ 
& \acs{X-Gluc} & 2\si{mM} &  \\ 
\begin{tabular}{
p{.29\textwidth}
p{.5\textwidth}
%p{1cm}
S[table-format=1.3,table-comparator=true,table-space-text-post={*****}]
p{.3\textwidth}}
&Der Puffer ist nur eingeschränkt lagerfähig und wurde daher unmittelbar vor Gebrauch angesetzt.&&\\
\end{tabular}\\
&  &  &  \\ 
\acs{HEPES} & \acs{HEPES} & 1\si{mM} &  \\ 
 & pH (NaOH) & 7,0 &  \\ 
\begin{tabular}{
p{.29\textwidth}
p{.5\textwidth}
%p{1cm}
S[table-format=1.3,table-comparator=true,table-space-text-post={*****}]
p{.3\textwidth}}
&Der Puffer wurde durch einen Filter mit 0,2\,$\upmu$m Porenweite sterilfiltriert und bei 4\celcius\ gelagert.&&\\
\end{tabular}\\
&&&\\
\acs{HEPES}-Puffer 1 & \acs{HEPES} & 1\si{\%}\ (v/v) &  \\ 
\begin{tabular}{
p{.29\textwidth}
p{.5\textwidth}
%p{1cm}
S[table-format=1.3,table-comparator=true,table-space-text-post={*****}]
p{.3\textwidth}}
&Der Puffer wurde durch einen Filter mit 0,2\,$\upmu$m Porenweite sterilfiltriert und bei 4\celcius\ gelagert.&&\\
\end{tabular}\\
&  &  &  \\ 
\acs{HEPES}-Puffer 2 & \acs{HEPES} & 1\si{\%}\ (v/v)&  \\ 
& Glyzerin & 10\si{\%}\ (v/v) &  \\ 
\begin{tabular}{
p{.29\textwidth}
p{.5\textwidth}
%p{1cm}
S[table-format=1.3,table-comparator=true,table-space-text-post={*****}]
p{.3\textwidth}}
&Der Puffer wurde durch einen Filter mit 0,2\,$\upmu$m Porenweite sterilfiltriert und bei 4\celcius\ gelagert.&&\\
\end{tabular}\\
&  &  &  \\ 
Infiltrationspuffer & \acs{MES} & 10\si{mM} &  \\ 
& MgCl$_2$ & 10\si{mM} &  \\ 
& Acetosyringon  & 200\si{$\upmu$M} &  \\ 
\begin{tabular}{
p{.29\textwidth}
p{.5\textwidth}
%p{1cm}
S[table-format=1.3,table-comparator=true,table-space-text-post={*****}]
p{.3\textwidth}}
&Der Puffer ist nur eingeschränkt lagerfähig und wurde daher unmittelbar vor Gebrauch angesetzt.&&\\
\end{tabular}\\
&  &  &  \\ 
KPO$_4$-Puffer & KH$_2$PO$_4$ & 0,17\si{M} &  \\ 
& K$_2$HPO$_4$ & 0,72\si{M} &  \\ 
\begin{tabular}{
p{.29\textwidth}
p{.5\textwidth}
%p{1cm}
S[table-format=1.3,table-comparator=true,table-space-text-post={*****}]
p{.3\textwidth}}
&Der Puffer wurde autoklaviert und bei Raumtemperatur gelagert.&&\\
\end{tabular}\\
&  &  &  \\ 
\acs{MES}&\acs{MES}&500\si{mM}&\\
&pH (NaOH) & 5,6&\\
\begin{tabular}{
p{.29\textwidth}
p{.5\textwidth}
%p{1cm}
S[table-format=1.3,table-comparator=true,table-space-text-post={*****}]
p{.3\textwidth}}
&Der Puffer wurde durch einen Filter mit 0,2\,$\upmu$m Porenweite sterilfiltriert und bei 4\celcius\ gelagert.&&\\
\end{tabular}\\
&&&\\
%\acs{MOPS} (10fach) & \acs{MOPS} & 400\si{mM} &  \\ 
%& Natriumacetat & 100\si{mM} &  \\ 
%& \acs{EDTA} & 10\si{mM} &  \\ 
%&  &  &  \\ 
Na$_2$HPO$_4$&Na$_2$HPO$_4$& 1\si{M}&\\
\begin{tabular}{
p{.29\textwidth}
p{.5\textwidth}
%p{1cm}
S[table-format=1.3,table-comparator=true,table-space-text-post={*****}]
p{.3\textwidth}}
&Die Lösung wurde autoklaviert und bei Raumtemperatur gelagert.&&\\
\end{tabular}\\
&&&\\
NaH$_2$PO$_4$ &NaH$_2$PO$_4$ & 1\si{M}&\\
\begin{tabular}{
p{.29\textwidth}
p{.5\textwidth}
%p{1cm}
S[table-format=1.3,table-comparator=true,table-space-text-post={*****}]
p{.3\textwidth}}
&Die Lösung wurde autoklaviert und bei Raumtemperatur gelagert.&&\\
\end{tabular}\\
&&&\\
NaPO$_4$-Puffer & Na$_2$HPO$_4$ (\si{1M}) & 72\si{\%}\ (v/v) &  \\ 
  &NaH$_2$PO$_4$ (\si{1M}) & 28\si{\%}\ (v/v) &  \\ 
 & pH & 7,2 &  \\ 
\begin{tabular}{
p{.29\textwidth}
p{.5\textwidth}
%p{1cm}
S[table-format=1.3,table-comparator=true,table-space-text-post={*****}]
p{.3\textwidth}}
&Der Puffer wurde autoklaviert und bei Raumtemperatur gelagert.&&\\
\end{tabular}\\
&&&\\
\newpage
PIPES & Pipes-Na$_2$&500\,mM&\\
						&	pH&6,7&\\
\begin{tabular}{
p{.29\textwidth}
p{.5\textwidth}
%p{1cm}
S[table-format=1.3,table-comparator=true,table-space-text-post={*****}]
p{.3\textwidth}}
&Der Puffer wurde durch einen Filter mit 0,2\,$\upmu$m Porenweite sterilfiltriert und bei 4\celcius\ gelagert.&&\\
\end{tabular}\\
&&&\\
\acs{TAE} (50fach) & \acs{TRIS} & 2\si{M} &  \\ 
 & Eisessig & 5,7\si{\%}\ (v/v) &  \\ 
 & \acs{EDTA} (0,5M) & 50\si{mM} &  \\ 
 & pH  & 8,0 &  \\ 
\begin{tabular}{
p{.29\textwidth}
p{.5\textwidth}
%p{1cm}
S[table-format=1.3,table-comparator=true,table-space-text-post={*****}]
p{.3\textwidth}}
&Zur Verwendung wurde der Puffer 1:50 mit \vewasser\ verdünnt.&&\\
\end{tabular}\\
&  &  &  \\ 
TB-Puffer& CaCl$_2$ & 15\,mM&\\
	&KCL&250\,mM&\\
	&PIPES&10\,mM&\\
	&\multicolumn{3}{l}{pH mit 1\,M KOH auf 6,7 einstellen}\\
&MnCl$_2$&55\,mM&\\
\begin{tabular}{
p{.29\textwidth}
p{.5\textwidth}
%p{1cm}
S[table-format=1.3,table-comparator=true,table-space-text-post={*****}]
p{.3\textwidth}}
&Der Puffer wurde durch einen Filter mit 0,2\,$\upmu$m Porenweite sterilfiltriert und bei 4\celcius\ gelagert.&&\\
\end{tabular}\\
&&&\\
\acs{TBE} (5fach) & \acs{TRIS} & 445\si{mM} &  \\ 
 & Borsäure & 445\si{mM} &  \\ 
 & \acs{EDTA}  & 10\si{mM} &  \\ 
\begin{tabular}{
p{.29\textwidth}
p{.5\textwidth}
%p{1cm}
S[table-format=1.3,table-comparator=true,table-space-text-post={*****}]
p{.3\textwidth}}
&Zur Verwendung wurde der Puffer 1:10 mit \vewasser\ verdünnt.&&\\
\end{tabular}\\
&  &  &  \\ 
\end{longtable}
\addtocounter{table}{-1}
\newpage
\section{Verbrauchsmaterialien}
\subsection{Kits}
Die in dieser Arbeit verwendeten Enzyme und Kits sowie die jeweiligen Hersteller sind in \Fref{tab:Enzyme} und \Fref{tab:Kits} aufgelistet.\\
\captionof{table}[Verwendeten Kits]{Liste der verwendeten Kits}
\label{tab:Kits}
\begin{tabular}{l
l
l
}
\toprule
Bezeichnung & Verwendung&Hersteller \\ 
\midrule
peqGold Cycle Pure Kit & PCR-Aufreinigung  & Peqlab GmbH, Erlangen\\  
peqGold Gel Extraction Kit & DNA Gelextraktion & Peqlab GmbH, Erlangen \\
peqGold Plasmid Mini Kit & Plasmidisolation & Peqlab GmbH, Erlangen\\
Plant RNA Isolation Kit & RNA Isolation & Agilent GmbH, Böblingen\\ 
Qubit RNA BR Assay Kit & RNA-Quantifizierung & Life Technologies,\\
&&Carlsbad, USA \\ 
Qubit DNA BR Assay Kit & DNA-Quantifizierung & Life Technologies,\\
& & Carlsbad, USA \\ 
Sensifast Probe no-ROX Kit & Real Time PCR & Bioline GmbH,\\ 
&&Luckenwalde\\
Sensifast Sybr no-ROX Kit & Real-Time PCR & Bioline GmbH, \\
&&Luckenwalde\\
Tetro cDNA Synthese Kit & cDNA-Synthese & Bioline GmbH, \\
&&Luckenwalde\\
\bottomrule
\end{tabular}
\vspace{12pt}
\subsection{Enzyme}
\captionof{table}[Verwendete Enzyme]{Liste der verwendeten Enzyme}
\label{tab:Enzyme}
\begin{tabular}{
l
l
l
}
\toprule
Bezeichnung & Enzym & Hersteller\\
\midrule
DNA-free &RNAse-freie DNAse& Life Technologies,\\
&&Carlsbad, CA, USA\\
FastAP & Alkalische Phosphatase & Fisher Scientific GmbH, Schwerte\\
FastDigest \BamHI & Restriktionsendonuklease & Fisher Scientific GmbH, Schwerte\\
FastDigest \EcoRI & Restriktionsendonuklease & Fisher Scientific GmbH, Schwerte\\
FastDigest \ClaI & Restriktionsendonuklease & Fisher Scientific GmbH, Schwerte\\
FastDigest \KpnI & Restriktionsendonuklease & Fisher Scientific GmbH, Schwerte\\
FastDigest \XbaI & Restriktionsendonuklease & Fisher Scientific GmbH, Schwerte\\
FastDigest \XhoI & Restriktionsendonuklease & Fisher Scientific GmbH, Schwerte\\
MightyMix & T4 DNA Ligase & Takara Bio Europe S.A.S,\\
 &&Saint-Germain-en-Laye, FR\\
Phusion &  DNA Polymerase & Fisher Scientific GmbH, Schwerte\\
\textit{Taq} Polymerase &  DNA Polymerase & Fisher Scientific GmbH, Schwerte\\
\bottomrule
\end{tabular}
%\vspace{12pt}\\
\subsection{Oligonukleotide}
Die Synthese von Oligonukleotiden für konventionelle \acs{PCR}-Anwendungen erfolgte bei der Biomers.net~GmbH, Ulm.
Oligonukleotide und \TaqMan\,-Sonden für real-time \acs{PCR}-Anwendungen waren  \acs{HPLC}-gereinigt und wurden von der Apara~Bioscience~GmbH, Denzlingen bezogen.
\subsection{Klonierungsprimer}
Die in dieser Arbeit verwendeten Klonierungsprimer sind in \Fref{tab:Kloprimer} aufgelistet. Diese Primer wurden mit 5'-Überhängen entworfen,  um DNA-Fragmente mit Restriktionsschnittstellen zu amplifizieren. Die so erzeugten PCR-Produkten konnten nach einem Restriktionsverdau in Plasmidvektoren ligiert werden. Zusätzlich zu den palindromischen Erkennungssequenzen wurden jeweils drei Schutzbasen angehängt.\\
\captionof{table}[Klonierungsprimer]{Liste der verwendeten Klonierungsprimer}
\label{tab:Kloprimer}
\setlength{\LTpre}{0pt}
\setlength{\LTpost}{0pt}
\small
\begin{longtable}{
p{.13\textwidth}
p{.57\textwidth}
p{.075\textwidth}
p{.12\textwidth}
}
\toprule
Bezeichnung & Überhang - Sequenz (5'$\rightarrow$3') & Tm\,(\celcius) & Schnittstelle\\
\midrule
\endhead
00415fw & TGAGGATCC-CACCTGCTCGTCCTAC & 70 & \BamHI \\ 
00415rv & TGAGGATCC-TGCTTACGCCGTTATATTGCC & 71 & \BamHI \\ 
01251fw & TGGGATCC-AAACTGTTGGCTTTTGATCCAT & 70 & \BamHI \\ 
01251rv & TGAGGATCC-TATCTGCCCCCTCATTTACACT & 71 & \BamHI \\ 
01371fw & CTGGGATCC-TGGCTTTTCTATCAGCAAGTGA & 71 & \BamHI \\ 
01371rv & TGAGGATCC-TCCCAGATCTAGTCCACCATCT & 72 & \BamHI \\ 
01750fw & TGAGGATCC-GCCTTCGCCAAGGAGACTTA & 72 & \BamHI \\ 
01750rv & TGAGGATCC-GGCAGTTGGCACATCAGTTG & 73 & \BamHI \\ 
04224fw & TGAGGATCC-CAGTCGTTGCCACCAAGTGT & 72 & \BamHI \\ 
04224rv & TGAGGATCC-CACGGCGACACCAATCATTA & 74 & \BamHI \\ 
04224\XhoI & CTACTCGAG-TTGCCACCAAGTGTACG & 68 & \XhoI \\ 
04224\KpnI & CTAGGTACC-TAGAGCACGGCGACACC & 71 & \KpnI \\ 
04224\XbaI & CTATCTAGAT-TGCCACCAAGTGTACG & 64 & \XbaI \\ 
04224\ClaI & CTAATCGAT-TAGAGCACGGCGACACC & 69 & \ClaI \\ 
05320fw & TGAGGATCC-GACTAGTGAAATATACCCTC & 66 & \BamHI \\ 
05320rv & TGAGGATCC-TCGCGTCTGTAAGCATCACT & 72 & \BamHI \\ 
06673fw & TGAGGATCC-CCTGGTTCCTTTGAACCACC & 73 & \BamHI \\ 
06673rv & TGAGGATCC-GATTTGATGTATTGATGTCTCTG & 67 & \BamHI \\ 
1976\XhoI & CTACTCGAG-GGTGCTAACTCTTC & 67 &\XhoI \\ 
1976\KpnI & CTAGGTACC-AGCCACAGTGACAATC & 66 & \KpnI \\ 
1976\XbaI & CTATCTAGA-GGTGCTAACACCTCTTC & 63 & \XbaI \\ 
1976\ClaI & CTAATCGAT-AGCCACAGTGACAATC & 63 & \ClaI \\ 
2356fw & TGAGGATCC-GGTGGGATGGGAACAGGTCGTAG & 76 & \BamHI \\ 
2356rv & TGAGGATCC-TGGTCTTGCAGTGGGAGTGATTC & 74 & \BamHI \\ 
2683\XhoI & CTACTCGAG-GTTGCTCAGTGAATAAGTC & 66 & \XhoI \\ 
2683\KpnI & CTAGGTACC-ATATGATACGAGAGGCTGTAG & 66 & \KpnI \\ 
2683\XbaI & CTATCTAGA-GTTGCTCAGTGAATAAGTC & 62 & \XbaI \\ 
2683\ClaI & CTAATCGAT-ATATGATACGAGAGGCTGTAG & 63 & \ClaI \\ 
3015fw & TGAGGATCC-GAGTTTGTAGACGGTCTGTCTGC & 73 & \BamHI \\ 
3015rv & TGAGGATCC-GAATAGAGCTTCCAGAGTCATCTG & 71 & \BamHI \\ 
462\XhoI & CTACTCGAG-GCAAAGGCTTGTATTAACG & 67 & \XhoI \\ 
462\KpnI & CTAGGTACC-GGCTCTAATTGTTTGTCAG & 66 & \KpnI \\ 
462\XbaI & CTATCTAGA-GCAAAGGCTTGTATTAACG & 63 & \XbaI \\ 
462\ClaI & CTAATCGAT-GGCTCTAATTGTTTGTCAG & 64 & \ClaI \\ 
PDS\XhoI & CTACTCGAG-AAAGAACAGCGCCTTCC & 68 & \XhoI \\ 
PDS\KpnI & CTAGGTACC-GCCCAAACCAGTCAATG & 69 & \KpnI \\ 
PDS\XbaI & CTATCTAGA-AAAGAACAGCGCCTTCC & 65 & \XbaI \\ 
PDS\ClaI & CTAATCGAT-GCCCAAACCAGTCAATG & 66 & \ClaI \\ 
iGUS\BamHI & CTAGGATCC-TCATTGTTTGCCTCCCTGCTGCGGT & 76 & \BamHI \\ 
iGUS\EcoRI & CTAGAATTC-ATGGTACGTCCTGTAGAAACCCCAA & 70 & \EcoRI \\ 
\bottomrule
\end{longtable}
\addtocounter{table}{-1}
\normalsize
\subsection{Primer und Sonden für real-time PCR-Anwendungen}
Die in \Fref{tab:realtimePrimer} aufgelisteten Primer und Sonden wurden zur Genexpressionsanalyse eingesetzt. Die Amplifikationseffizienz (E) der einzelnen Primerpaare wurde über Standardkurven bestimmt (Siehe MM Standardkurven, Siehe Anhang Stdcrv). Die verwendete \TaqMan -Sonde war mit dem Fluorophor \acs{FAM} und dem Quencher \acs{TAMRA} gelabelt.\\ 
\captionof{table}[Primer und Sonden für real-time PCR-Anwendungen]{Liste der verwendeten Primer und Sonden für real-time PCR Anwendungen}
\label{tab:realtimePrimer}
\setlength{\LTpre}{0pt}
\setlength{\LTpost}{0pt}
\small
\begin{longtable}{
L{.175\textwidth}
p{.55\textwidth}
p{.085\textwidth}
p{.055\textwidth}
}
\toprule
Bezeichnung & Sequenz 5' $\rightarrow$ 3' & Tm\,(\celcius)&E\,(\%) \\ 
\midrule
\endhead 
ActinDis1f  & ACAGTTTCACCACAACCGCC & 65 &\\ 
ActinDis1r  & TGACCGTCGGGAAGTTCG & 63 &\\ 
AtubDis1f  & CTGCGAACAACTATGCTCGTC & 63& \\ 
AtubDis1r & CACGAAGAAGCCTTGGAGTCC & 64 &\\ 
CytB1f  & TCAAGACGCATCCAAATTCTAGGTC & 64& \\ 
CytB1r  & GTGTTACACCCGTGATAATCTGAATGAT & 65& \\ 
Elf1a1f  & GTGAGCGTGGTATCACCATC & 62& \\ 
Elf1a1r  & CAGAATGGCGCAATCAGC & 61& \\ 
Elf1a2f  & GGAAATGGATACGCTCCTGTC & 62& \\ 
Elf1a2r  & CTTAACTAAGGCGGCGTCTC & 62 &\\ 
GAPDH1f  & GGTATGGCTTTCCGAGTTCCA & 64& \\ 
GAPDH1r  & TCAGTTGATACCAAATCATCCTCAG & 62 &\\ 
Gmcons4fw  & GATCAGCAATTATGCACAACG & 60 &\\ 
Gmcons4rv  & CCGCCACCATTCAGATTATGT & 62 &\\ 
Gmcons6fw  & AGATAGGGAAATGGTGCAGGT & 63 &\\ 
Gmcons6rv  & CTAATGGCAATTGCAGCTCTC & 61 &\\ 
Gmcons7fw  & ATGAATGACGGTTCCCATGTA & 61 &\\ 
Gmcons7rv  & GGCATTAAGGCAGCTCACTCT & 64& \\ 
Gmcons15fw  & TAAAGAGCACCATGCCTATCC & 61& \\ 
Gmcons15rv  & TGGTTATGTGAGCAGATGCAA & 62 &\\ 
RibPro2f & CGGCAACAGTTGTATGACCTC & 63 &\\ 
RibPro2r & AGTGTCAGCCTCAGATCTTGG & 63 &\\ 
RibPro3f & GTGAATGGGAGACCAATCTCAG & 62 &\\ 
RibPro3r & TTGCCTCCTCCATGAGTCAG & 63 &\\ 
Ubc1f & CGGACCAGTACCCTTACAAATC & 62 &\\ 
Ubc1r & ATCAAACATCGGCGACCAG & 62 &\\ 
UbcE22f & ATATACCCTAACCCGGAGTCG & 62 &\\ 
UbcE22r & GTTCCTGGCATGGATATCAGTC & 62 &\\ 
UbcE23f & GTCGAACTGTGACGAGTTTG & 61 &\\ 
UbcE23r & ACGGCCTTAGTCTTCGATG & 61 &\\ 
q00153fw & AGTTGATCGAGTGACTGGTG & 61& \\ 
q00153rv & CATCTTGGGCAGCCAACATG & 63 &\\ 
q00239fw & GCGGAAAAGGATAAGGGG & 59 &\\ 
q00239rv & TCCGATCCTTAGTCTGGCCT & 64 &\\ 
q00241fw & CAATCGCCTGAGGACCGTAA & 63 &\\ 
q00241rv & CTGGGGCAACTTGTAGAGCA & 64 &\\ 
q00415Fw & CGAGAGTGTGCTGAAGCAGT & 64 &\\ 
q00415Rv & TCCTCAATTCCCAGGAGGTCT & 64 &\\ 
q00583fw & AATGCGTGGTCTCTCTGGTG & 64 &\\ 
q00583rv & GCTCGTCCAAGATCACCACA & 64 &\\ 
q00682fw & GGACTGGGCTTCAAGACTCC & 64 &\\ 
q00682rv & GAATCCTGCCCCTGATCGAG & 64 &\\ 
q01371fw & TGCCACTGGAGCAAAATCAC & 63 &\\ 
q01371rv & AGTGGAACTAAGCAGGGAGG & 62 &\\ 
q01750fw & ATGTGGTGAATGGGTGAGGC & 64 &\\ 
q01750rv & CTTTCGAGGGGCCCAGATTC & 64 &\\ 
q02726fw & ACCTCCCGTTCAGCTAGTCT & 64 &\\ 
q02726rv & AATTCATCAGAGTCGGCCCC & 64 &\\ 
q04224F1 & CCTAAGAGGTTTGAGTTAGCTG & 60& \\ 
q04224R1 & CTGCAAAGATGATTTGCCTCTC & 61& \\ 
q05106fw & CTTCGTGCCGCTTTGTGATT & 63 &\\ 
q05106rv & GGGGTTTGTCGTCGGTTTTG & 64& \\ 
q05320fw & GTTGCTTGCATTGGAACGTT & 62 &\\ 
q05320rv & TTTACAACGTTGCTGGCCAC & 63 &\\ 
q05320as-R1 & TCGACGGTCTTGAAGAGTGA & 62& \\ 
q1976Fw & TGCAGCATTGGTTTTGGGCG & 66& \\ 
q1976Rv & AGGTTGCTGAGCCGCTTGTT & 66 &\\ 
q2356Fw & TAAACAGACCGCAGTGGTGG & 64 &\\ 
q2356Rv & CCTCGTTGTAGCCTGGTTGT & 64 &\\ 
q2683Rv & TGGAACACAGTTTTGGGCAGT & 64 &\\ 
q3015Fw & TCCAGCTATCGCCAACAACC & 64 &\\ 
q3015Rv & TCCACAGTTCCTCCTCCGTC & 65 &\\ 
q3015as-R1 & CGACACAGATTGTGATGGAA & 59& \\ 
q462Fw & CCGGCGCATACACCAACTCA & 66 &\\ 
q462Rv & GCGTCCAAAGCCCATAGTGC & 64 &\\ 
pBPMV-F1 & ACATTCCTGGGAATTGATCTTCC & 63& \\ 
pBPMV-R1 & GATCGGGGAAATTCGAGCTATC & 59 &\\ 
qBPMV-Probe&\footnotesize{\acs{FAM}-}\small{TCCTCATGCAGAGGATTCCGCA}\footnotesize{-\acs{TAMRA}}&69&\\
qGUS-Fw & CTGGGTGGACGATATCACCG & 64& \\ 
qGUS-Rv & TCCAGTTGCAACCACCTGTT & 64& \\ 
qPDK-Fw & TGTTAGAAATTCCAATCTGCTTGT & 60& \\ 
qPDK-Rv & AATGATAGATCTTGCGCTTTGTT & 61& \\ 
\bottomrule
\end{longtable}
%\par\medskip 
\footnotesize
\textbf{$^a$} \citep{Schmitz.2013},\textbf{$^b$}\citep{Libault.2008}
\normalsize
\section{Biologisches Material}
%%%
%%%
%%%
\subsection{Saatgut und Anzucht von \Gmax}\label{sec:Saatgut}
Für die Anzucht von Sojabohnen (\textit{Glycine max} (L.) Merr) wurde Saatgut der Sorte Thorne (Bayer CropScience AG, Lyon, Frankreich) verwendet. Die Kultivierung erfolgte ohne Düngung in Topfsubstrat (Einheitserde Typ T, Gebr. Patzer GmbH, Sinntal-Jossa) bei einer Tag/Nacht-Periode von 16\,h/8\,h und 22$^\circ$\,C Umgebungstemperatur. 
%%%
%%%
%%%
\subsection{Pilzisolat}
Im Rahmen der vorliegenden Arbeit wurden Uredosporen eines kompatiblen Wildisolats (Thai~1) des Asiatischen Sojabohnenrostes \textit{ P. pachyrhizi} \,Syd.\,\&\,P.Syd aus der Stammsammlung des Instituts für Phytomedizin, Universität Hohenheim verwendet.
%%%
%%%
%%%
\subsubsection{Inokulation von \textit{G.\,max} mit \textit{P.\,pachyrhizi}} 
Zur Inokulation von \Gmax\ mit \Ppach\ wurden die Blätter 21-tage alter Sojabohnen gleichmäßig mittels eines DC-Zerstäubers (Carl Roth GmbH, Karlsruhe) mit \SI{0,002}{\%} (w/v) Inokulationssuspension besprüht und anschließend bei Dunkelheit, 95\% relativer Luftfeuchte und 20$^\circ$\,C für 12\,h inkubiert. Die weitere Kultivierung der Pflanzen erfolgte unter den in \ref{sec:Saatgut} beschriebenen Bedingungen.
%%%
%%%
%%%
\subsubsection{\textit{In vitro}-Erzeugung von Keimschläuchen}\label{sec:Keimschlauch}
Die \textit{in vitro}-Erzeugung von Keimschläuchen von \Ppach\ erfolgte nach der von \citet{PosadaBuitrago.2005} beschriebenen Methode. Dafür wurden 100\,mg tiefgefrorene Uredosporen für 5\,min einem Hitzeschock bei 42\,\celcius\ unterzogen und anschließend gleichmäßig auf die Wasseroberfläche einer mit \acs{\vewasser}\ gefüllten Petrischale verteilt. Zur Keimung wurden die Uredopsoren für 12\,h bei Raumtemperatur und Dunkelheit inkubiert.
%%%
%%%
%%%
\subsubsection{\textit{In vitro}-Erzeugung von  Appressorien }\label{sec:appressorien}
Zur \textit{in vitro}-Erzeugung von Appressorien wurden kreisrunde Stücke \acs{PE}-Folie (\O\,20\,cm)  gleichmäßig mittels eines DC-Zerstäubers (Carl Roth GmbH, Karlsruhe) mit 0,002\,\% Uredosporensuspension besprüht und in Glaspetrischalen für 16\,h bei Raumtemperatur und Dunkelheit inkubiert. 
%%%
%%%
%%%
\subsection{Bakterienstämme}
Die Vermehrung von Plasmidkonstrukten erfolgte in \textit{Escherichia coli} DH\,10B \citep{Grant.1990}.
Für die transiente Transformation von \textit{G. max} und \textit{N. benthamiana} wurde \textit{A. tumefaciens } LBA\,4404 \citep{Ooms.1981} verwendet. 
%%%%%%
%%%%%%
%%%%%%
\subsubsection{Herstellung \acs{SEM}-kompetenter Zellen von \Ecoli }
Die Herstellung \acs{SEM}-kompetenter Zellen erfolgte nach der von \citet{Inoue.1990} beschriebenen Methode.
Zur Herstellung einer Vorkultur wurden 5\,ml LB-Medium mit einer Kolonie \Ecoli\ DH10B angeimpft und über Nacht bei 37\celcius\ und 125\,\acs{rpm} auf einem Rotator inkubiert. Am Folgetag wurden 250\,ml SOB-Medium mit 2\,\%\,(v/v) Vorkultur angeimpft und bis zum erreichen einer \acs{OD$_{600}$}\,=\,0,6 auf einem Schüttler bei Raumtemperatur inkubiert. Die Zellen wurden für 10\,min auf Eis inkubiert und anschließend in vorgekühlte 250\,ml Zentrifugenbecher überführt. Es folgte eine 10-minütige Zentrifugation bei 4\celcius\ und 2500\,\acs{rcf}. Der Überstand wurde verworfen und die Zellen in 80\,ml eiskaltem TB-Puffer resuspendiert. Die Zellen wurden für 10\,min auf Eis inkubiert und anschließend erneut für 10\,min bei 4\celcius\ und 2500\,\acs{rcf} zentrifugiert. Der Überstand wurde verworfen und die Zellen in 20\,ml eiskaltem TB-Puffer resuspendiert. Es wurde \acs{DMSO} zu einer Endkonzentration von 7\%\,(v/v) zugegeben und die Zellen erneut für 10\,min auf Eis inkubiert. Anschließend wurden die Zellen zu je 100\,$\upmu$l in sterile 2\,ml Reaktionsgefäße aliquotiert und in flüssigem Stickstoff schockgefroren. Die Lagerung der kompetenten Zellen erfolgte bei -70\celcius.      
\subsubsection{Transformation von \Ecoli}
Die Transformation SEM-kompetenter Zellen von \Ecoli\ erfolgte mittels Hitzeschock.
Dafür wurden 100\,$\upmu$l SEM\,-\,kompetente Zellen auf Eis aufgetaut und 50\,-\,100\,ng Plasmid-DNA dazupipettiert. Der Transformationsansatz wurde 30\,min auf Eis inkubiert und anschließend einem Hitzeschock (60\,s, 42\celcius) unterzogen. Nach einer 5-minütigen Inkubation auf Eis, wurde 1\,ml SOC-Medium dazugegeben und behutsam auf-und abpipettiert. Anschließend wurde die Zellsuspension für 2\,h bei 37\celcius\ und 125\,rpm auf einem Rotator inkubiert. Abschließend wurden 200\,\,$\upmu$l des Transformationsansatzes auf Selektivmedium ausplattiert und für 12\,-\,16\,h bei 37\celcius\ inkubiert. 
%%%
%%%
%%%
\subsubsection{Herstellung elektro-kompetenter Zellen von \Atumefaciens}
Die Herstellung elektro-kompetenter Zellen von \Atumefaciens\ erfolgte nach einer modifizierten Variante der von  \citet{Seidman.2001} beschriebenen Methode. 
Zur Herstellung einer Vorkultur, wurden 5\,ml YEB-Medium$_{Rif}$ mit einer Kolonie von \Atumefaciens\ LBA 4404 angeimpft und über Nacht bei 28\celcius\ und 125\,\acs{rpm} auf einem Rotator inkubiert. Am Folgetag wurden 200\,ml YEB-Medium$_{Rif}$ mit 2 \%\,(v/v) Vorkultur angeimpft und in einem 1000\,ml Erlenmeyerkolben bei 28\celcius\, und 150\,\acs{rpm}  bis zum Erreichen einer \acs{OD$_{600}$}\,=\,0,6 auf einem Schüttler inkubiert. Die Zellen wurden für 15\,min in einem Eiswasserbad inkubiert und anschließend in vorgekühlte 250\,ml Zentrifugenbecher überführt. Es folgte eine 20-minütige Zentrifugation bei 4\celcius\ und 1900\,\acs{rcf}. Der Überstand wurde verworfen und die pelletierten Zellen in 20\,ml eiskaltem HEPES-Puffer 1 resuspendiert. Die Zellen wurden erneut für 20\,min bei 4\celcius\ und 1900\,\acs{rcf} zentrifugiert und nach Verwerfen des Überstandes in 1\,ml eiskaltem HEPES-Puffer 2 resuspendiert. Die resuspendierten Zellen wurden in 2\,ml Reaktionsgefäße überführt und 5\,min bei 4\celcius\ und 10000\,\acs{rcf} zentrifugiert. Der Überstand wurde verworfen und die pelletierten Bakterien in 200\,$\upmu$l eiskaltem HEPES-Puffer 2 resuspendiert. Die resuspendierten Bakterien wurden zu 50\,$\upmu$l in 2\,ml Reaktionsgefäße aliquotiert und in flüssigem Stickstoff schockgefroren. Die Lagerung der kompetenten Zellen erfolgte bei -70\celcius.      
%%%
%%%
%%%
\subsubsection{Transformation von \Atumefaciens}
Die Elektroporation von \Atumefaciens\ erfolgte mit einem ECM 600 Electro Cell Manipulator (BTX Harvard Apparatus, Holliston, MA, USA). Dafür wurden 50\,$\upmu$l kompetente Zellen auf Eis aufgetaut und 25\,-\,50\,ng Plasmid-DNA dazupipettiert. Die Zellen wurde für 5\,min auf Eis inkubiert und anschließend in eine gekühlte Elektroporationsküvette ( Peqlab GmbH, Erlangen) mit 2\,mm Elektrodenabstand überführt. Die Elektroporation erfolgte bei 2,5\,kV, 200\,$\Omega$ und 25\,$\upmu$F. In die Elektroporationsküvette wurde 1\,ml SOC-Medium gegeben und behutsam auf- und abpipettiert. Anschließend wurde die Zellsuspension in ein steriles 2\,ml Reaktionsgefäß überführt und für 2\,h bei 28\celcius\ und 125\,rpm auf einem Rotator inkubiert. Abschließend wurden 200\,\,$\upmu$l des Transformationsansatzes auf Selektivmedium ausplattiert und für 48-72\,h bei 28\celcius\ inkubiert. 
\section{Geräte}
\begin{longtable}{
L{.3\textwidth}
L{.3\textwidth}
L{.3\textwidth}
}
\toprule
Gerät & Bezeichnung, Hersteller & Verwendung \\ 
\midrule
\endhead
 Biolistik-System &  PDS-100-He$\textsuperscript{\textregistered}$, BioRad GmbH, München & Transformation von \Gmax \\ 
 Elektro-Zell-Manipulator &  ECM 600$\textsuperscript{\textregistered}$, BTX Harvard Apparatus, Holliston, USA & Transformation von \Atumefaciens \\ 
 Fluorometer & Qubit 2.0$\textsuperscript{\textregistered}$, Life~Technologies GmbH, Darmstadt & Quantifizierung von Nukleinsäuren \\
Geldokumentations-system &  Quantum 1100$\textsuperscript{\textregistered}$, PEQLAB GmbH, Erlangen & Auswertung von Agarosegelen  \\ 
 Gelelektrophorese-kammer &  wissenschaftliche Werkstätten, Universität Konstanz &  Gelelektrophorese \\ 
Homogenisator& FastPrep$\textsuperscript{\textregistered}$-24, MP~Biomedicals GmbH, Eschwege & Aufschluss biologischer Materialien\\
Mikroskop &  Primo Star, Zeiss AG &  Mikroskopie\\ 
Orbitalschüttler &  Shaker DOS 10L, LTF Labortechnik & Inkubation von Bakterienkulturen \\ 
PCR-Cycler &  C1000 touch, BioRad GmbH, München &  PCR \\ 
 &  Cfx96, BioRad GmbH, München & real-time PCR \\ 
 &  Masterycler gradient, Eppendorf AG, Hamburg &  PCR \\ 
 Rotator & Rotator, NeoLab GmbH, Heidelberg &  Inkubation von Bakterienkulturen \\ 
 Thermoblock &  Thriller, PEQLAB GmbH, Erlangen &  Inkubation von Reaktionsansätzen \\ 
 Vortexmischer &  VM-300, neolab GmbH, Heidelberg & Mischen von Reaktionsansätzen \\ 
Wasserbad &  F12, Julabo GmbH, Seelbach &  Inkubation von Reaktionsansätzen \\ 
 Zentrifugen &  Sorvall RC5B, DuPont &  Ultrazentrifugation \\ 
 &  5417R, Eppendorf AG, Hamburg & Zentrifugation \\ 
 &  5415R, Eppendorf AG, Hamburg &Zentrifugation \\ 
\bottomrule
\end{longtable}

\section{Software und Server}


\section{Plasmide}
Als Ausgangsplasmid für den Aufbau viraler Silencingkonstrukte diente pBPMV-IA-V1 (Quelle). Für den Aufbau  hpRNA-exprimierender Genkonstrukte wurde das Plasmid pHannibal (Quelle) verwendet. Zur Transformation von \textit{G. max} und \textit{N. benthamiana} mit den in pHannibal aufgenauten Konstrukten,  wurde das binäre Plasmid pART27 verwendet  

\subsubsection*{pBPMV}

\subsubsection*{pHannibal und pART27}
\begin{center}
%\begin{center}
\includegraphics[scale=.3]{MaterialMethoden/Abb/pHannibalgb1}
\end{center}
\captionof{figure}[Plasmid pHannibal]{Plasmid pHannibal für den Aufbau von hpRNA-Konstrukten\\ \footnotesize blablabla}
\par


\section{Molekularbiologische Methoden}
\subsection{RNA-Präparation}
\section{Isolierung von Nukleinsäuren}
\subsection{Isolation von Plasmid DNA}
Die Isolation von Plasmid-DNA aus \Ecoli\ und \Atumefaciens\ erfolgte mittels peqGold Plasmid Mini Kit (Peqlab GmbH, Erlangen) nach den Angaben des Herstellers. 
\subsection{Isolation von RNA aus Pflanzenmaterial}\label{sec:RNA-Pflanze}
Die Isolation von RNA aus infiziertem und nicht-infiziertem Pflanzenmaterial von \Gmax\ erfolgte nach einem modifizierten Protokoll mittels Plant RNA Isolation Kit (Agilent GmbH, Böblingen). Dafür wurden bis zu 100\, mg Pflanzenmaterial mit einem Korkbohrer ausgestanzt und zusammen mit zwei Edelstahlkügelchen (\O 4\,mm) in 2\, ml Schraubdeckelröhrchen überführt. Das Pflanzenmaterial wurde in flüssigem Stickstoff schockgefroren und zweimal für 20\,s bei 4000 \acs{mps} im FastPrep$\textsuperscript{\textregistered}$-24 homogenisiert, wobei das Material zwischen den Homogenisierungsschritten erneut schockgefroren wurde. Anschließend wurden 600\,$\upmu$ Extraktionslösung dazugegeben und durch vortexen gemischt. Das Homogenat wurde für 2\,min bei 4\celcius\ und 16.000\,rpm zentrifugiert. Anschließend wurde der Überstand abgenommen und auf ein Filtersäulchen überführt. Nach einer 3-minütigen Zentrifugation bei 4\celcius\ und und 16.000\,rpm wurde der Durchfluß in ein RNAse-freies 2\,ml Reaktionsgefäß überführt und 600\,$\upmu$l Isopropanol dazugegeben. Die Lösung wurde durch mehrfaches invertieren gemischt und für 5\,min bei Raumtemperatur inkubiert. Anschließend wurden 600\,$\upmu$l der Lösung auf ein Isolationssäulchen überführt und für 30\,s bei  4\celcius\ und 16.000\,rpm zentrifugiert. Der Durchfluss wurde verworfen und das Säulchen erneut mit 600 $\upmu$l beladen und ein weiteres mal für 30\,s bei  4\celcius\ und 16.000\,rpm zentrifugiert. Es folgten 2 Waschschritte bei welchen jeweils 500\,$\upmu$l Waschlösung auf das Säulchen gegebebn wurden und anschließend für 30 s bei  4\celcius\ und 16.000\,rpm zentrifugiert wurde. Der Durchfluss wurde verworfen und das Säulchen zur Trocknung der gebundenen RNA für 2\,min bei 4\celcius\ und 16.000\,rpm zentrifugiert. Anschließend wurden 30-50\,$\upmu$l \depcwasser\ auf die Mitte der Säulchenmembran pipettiert und das Säulchen auf ein RNAse-freies 1,5\,ml Reaktionsgefäß überführt. Nach einer Inkubationszeit von 2\,min bei Raumtemperatur wurde die gelöste RNA durch Zentrifugation für 30 s bei  4\celcius\ und 16.000\,rpm in das Reaktionsgefäß überführt. Die kurzzeitige Lagerung der RNA erfolgte auf Eis. Zur langfristigen Lagerung wurde die RNA durch Zugabe von 1/10 Volumen Natriumacetat (3M, pH irgendwas) und 2,5 Volumen EtOH gefällt und bei -80\celcius\ aufbewahrt. 
\subsection{Isolation von RNA aus Uredosporen und Keimschläuchen von \Ppach}
Die Isolation von RNA aus Uredosporen und Keimschläuchen (siehe \ref{sec:Keimschlauch}) erfolgte analog zu der in \ref{sec:RNA-Pflanze} beschriebenen Methode.    
\subsection{Isolation von RNA aus Appressorien von \Ppach}
Zur Isolation von RNA aus Appressorien von \Ppach\ wurden die in\ref{sec:appressorien} beschriebenen PE-Folie behutsam mit Filterpapier trocken getupft und anschließend 600\,$\upmu$l Extraktionslösung darauf gegebenen. Die Extraktionslösung wurde mit einem Gummiwischer (Carl Roth GmbH, Karlsruhe) verteilt, wodurch sich die pilzlichen Strukturen von der Folie lösten und mit einer abgeschnittenen Pipettenspitze in ein 2\,ml Reaktionsgefäß mit einer Mischung aus Quarzsand und Glaskügelchen (Lysing Matrix E, MP Biomedicals GmbH, Eschwege) überführt werden konnten. Die Appressorien wurden 2-mal für 20\,s bei 6500 \acs{mps} im FastPrep$\textsuperscript{\textregistered}$-24 homogenisiert, wobei das Material zwischen den Homogenisierungsschritten auf Eis gekühlt wurde. Das Homogenat wurde für 2\,min bei 4\celcius\ und 16.000\,rpm zentrifugiert. Anschließend wurde der Überstand abgenommen und auf ein Filtersäulchen überführt. Die weitere Vorgehensweise erfolgte analog zu der in \ref{sec:RNA-Pflanze} beschriebenen Methode. 
\bibliographystyle{DissLit}
\bibliography{Literatur}
\end{document}